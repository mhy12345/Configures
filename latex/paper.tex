\documentclass[a4paper, 11pt]{article}

%%%%%% 导入包 %%%%%%
\usepackage{xeCJK}
\usepackage{graphicx}
\usepackage[unicode]{hyperref}
\usepackage{xcolor}
\usepackage{cite}
\usepackage{indentfirst}
\usepackage{booktabs}				%支持表格

\setmainfont{Times New Roman}
\setCJKmainfont[BoldFont={黑体},ItalicFont={楷体-简}]{黑体-简 细体}
\setsansfont{黑体}


%%%%%% 设置字号 %%%%%%
\newcommand{\chuhao}{\fontsize{42pt}{\baselineskip}\selectfont}
\newcommand{\xiaochuhao}{\fontsize{36pt}{\baselineskip}\selectfont}
\newcommand{\yihao}{\fontsize{28pt}{\baselineskip}\selectfont}
\newcommand{\xiaoyihao}{\fontsize{25pt}{\baselineskip}\selectfont}
\newcommand{\erhao}{\fontsize{21pt}{\baselineskip}\selectfont}
\newcommand{\xiaoerhao}{\fontsize{18pt}{\baselineskip}\selectfont}
\newcommand{\sanhao}{\fontsize{15.75pt}{\baselineskip}\selectfont}
\newcommand{\sihao}{\fontsize{14pt}{\baselineskip}\selectfont}
\newcommand{\xiaosihao}{\fontsize{12pt}{\baselineskip}\selectfont}
\newcommand{\wuhao}{\fontsize{10.5pt}{\baselineskip}\selectfont}
\newcommand{\xiaowuhao}{\fontsize{9pt}{\baselineskip}\selectfont}
\newcommand{\liuhao}{\fontsize{7.875pt}{\baselineskip}\selectfont}
\newcommand{\qihao}{\fontsize{5.25pt}{\baselineskip}\selectfont}

%%%% 设置 section 属性 %%%%
\makeatletter
\renewcommand\section{\@startsection{section}{1}{\z@}%
{-1.5ex \@plus -.5ex \@minus -.2ex}%
{.5ex \@plus .1ex}%
{\normalfont\sanhao\CJKfamily{hei}}}
\makeatother

%%%% 设置 subsection 属性 %%%%
\makeatletter
\renewcommand\subsection{\@startsection{subsection}{1}{\z@}%
{-1.25ex \@plus -.5ex \@minus -.2ex}%
{.4ex \@plus .1ex}%
{\normalfont\sihao\CJKfamily{hei}}}
\makeatother

%%%% 设置 subsubsection 属性 %%%%
\makeatletter
\renewcommand\subsubsection{\@startsection{subsubsection}{1}{\z@}%
{-1ex \@plus -.5ex \@minus -.2ex}%
{.3ex \@plus .1ex}%
{\normalfont\xiaosihao\CJKfamily{hei}}}
\makeatother

%%%% 段落首行缩进两个字 %%%%
\makeatletter
\let\@afterindentfalse\@afterindenttrue
\@afterindenttrue
\makeatother
\setlength{\parindent}{2em}  %中文缩进两个汉字位


%%%% 下面的命令重定义页面边距,使其符合中文刊物习惯 %%%%
\addtolength{\topmargin}{-54pt}
\setlength{\oddsidemargin}{0.63cm}  % 3.17cm - 1 inch
\setlength{\evensidemargin}{\oddsidemargin}
\setlength{\textwidth}{14.66cm}
\setlength{\textheight}{24.00cm}    % 24.62

%%%% 下面的命令设置行间距与段落间距 %%%%
\linespread{1.4}
% \setlength{\parskip}{1ex}
\setlength{\parskip}{0.5\baselineskip}

%%%% 定理类环境的定义 %%%%
\newtheorem{example}{例}             % 整体编号
\newtheorem{algorithm}{算法}
\newtheorem{theorem}{定理}[section]  % 按 section 编号
\newtheorem{definition}{定义}
\newtheorem{axiom}{公理}
\newtheorem{property}{性质}
\newtheorem{proposition}{命题}
\newtheorem{lemma}{引理}
\newtheorem{corollary}{推论}
\newtheorem{remark}{注解}
\newtheorem{condition}{条件}
\newtheorem{conclusion}{结论}
\newtheorem{assumption}{假设}

%%%% 重定义 %%%%
\renewcommand{\contentsname}{目录}  % 将Contents改为目录
\renewcommand{\abstractname}{摘要}  % 将Abstract改为摘要
\renewcommand{\refname}{参考文献}   % 将References改为参考文献
\renewcommand{\indexname}{索引}
\renewcommand{\figurename}{图}
\renewcommand{\tablename}{表}
\renewcommand{\appendixname}{附录}
\renewcommand{\algorithm}{算法}

%%%% 定义标题格式,包括title,author,affiliation,email等 %%%%
\title{\xiaoyihao 面向对象程序设计个人大作业\\ \erhao 基于SAT的路径匹配问题}
\author{毛晗扬 \footnote{电子邮件: maohy16@mails.tsinghua.edu.cn,学号: 2016011275}\\[2ex]
\xiaosihao 清华大学计算机系 计62\\[2ex]
}
\date{2017年5月}

%%%% 正文开始 %%%%
\usepackage{wrapfig}
\begin{document}
\maketitle
\newpage
\tableofcontents
\newpage

\section{问题概述}
\begin{wrapfigure}{r}{0.4\textwidth}
	\includegraphics[width=0.35\textwidth]{game.png} 
\end{wrapfigure}
一个N*M的方格图中,有一些不同颜色的终端,每种颜色恰好有两个终端.同时在一些位置存在“不可布线点”.现在需要生成一个合理的布线方案,使得相同颜色的两个终端两两相连,并且布线不想交,不经过不可布线点.

生成的布线方案应该依次满足:

\begin{itemize}
	\item 连通的终端对数尽可能大
	\item 所有布线位置的总数尽可能小(及总线长尽可能小)
\end{itemize}

\section{数学建模}

\subsection{用转化为布尔可满足性问题}
本问题可以通过数学建模转化为一个SAT问题.对于一种布线方案,我们可以将其连接的两个终端随意选择一个作为该布线的起点位置.

令$st_{w}$表示第w种布线的两个终端中我们选择的起点.

令 $p(i,j)$ 代表地图上的每一个位置$(i,j)$;$w$代表第$w$种布线;$t$代表离路径起点距离$t$(也可理解为从起点可以经过布线在$t$时刻到达某个位置).


同时,令

$$
c^{t}_{p,w}
$$

表示从$w$颜色的两个终端的起点终端出发,在$t$时刻能否到达$p$位置.这是一个Boolean数组,最为边界,$t^*$表示t能取到的最大值

在这种定义下,对于每一种可能的解,我们都可以找到一组$c^t_{p,w}$的构造方式以满足.同理,如果我们要求出一个解,我们就需要解决满足题目限制的布尔可满足性问题.

\subsection{限制条件}
\subsubsection{位置可达条件}
如果一个位置$c^t_{p,w}$确实在t时刻可以从$st_w$到达p位置,$c^t_{p,w}$才可能为真.\footnote{$N_5(p)$表示位置p及其周围的四个位置}

$$
c^t_{p,w} \Rightarrow \bigvee c^{t-1}_{N_5(p),w} 
$$



同时,作为边界条件,我们需要限制$c^0_{p,w}$,以及$c^{t^*}_{p,w}$的值,令$r_{p}$表示我们选择连接终端p
$$
r_{p} \Rightarrow c^0_{p,w} \land c^{t^*}_{p,w}
$$

\subsubsection{路径相交条件}

令$d_{p,w}$表示位置$p$已经被$w$标号的终端的连线覆盖,则显然不能被其他$w'$覆盖
$$
d_{p,w} = \bigvee_t c^t_{p,w}
$$

则

$$
\forall_p \forall_{w_1} \forall_{w_2} \overline{d_{p,w_1} \land d_{p,w_2}}
$$

\subsubsection{最优化条件}
我们分别需要最优化连接终端对数以及被占位置总数

$$
\mathbf{maximize} \ tot\_path = \sum_{w_0 \in w} to\_int(r_{w_0})
$$

$$
\mathbf{mimimize} \ tot\_length = \sum_{p_0 \in p} to\_int(\bigvee_{t,w} c^t_{p,w})
$$

\subsection{实现细节}
Z3是微软研究出来的一个公理推演系统,可以判断若干命题的可满足性和最优化问题。恰好可以用来解决这个问题。

详细代码架构可以参见代码README.md

\subsubsection{基于二分}
我们可以将优化条件改为二分判断可行性。

首先二分$tot\_path$,设当前二分值$v$,则添加限制条件

$$
tot\_path > v
$$

判断是否有满足条件的解


\subsubsection{基于Z3-Optimize}
Z3有专门的最大化、最小化单个变量的函数,可以直接调用该函数得到最优解,实际sat模型建立和之前相同。

\section{运行表现}
\subsection{算法比较}
对基于二分和基于Optimize的两个算法进行比较,结果如下表

\begin{table}[!htp]
	\centering
	\begin{tabular}{@{}lllll@{}}
		\toprule
		行数 & 列数 & 终端对数 & 算法 & 时间 \\ \midrule
		3 & 3 & 3 & 内置优化 & 0min0s96ms \\
		3 & 3 & 3 & 二分 & 0min0s9ms \\
		4 & 4 & 4 & 内置优化 & 0min0s338ms \\
		4 & 4 & 4 & 二分 & 0min0s28ms \\
		5 & 5 & 5 & 内置优化 & 0min0s839ms \\
		5 & 5 & 5 & 二分 & 0min0s61ms \\
		6 & 6 & 6 & 内置优化 & 0min2s36ms \\
		6 & 6 & 6 & 二分 & 0min0s143ms \\
		7 & 7 & 7 & 内置优化 & 0min5s549ms \\
		7 & 7 & 7 & 二分 & 0min0s303ms \\
		8 & 8 & 8 & 内置优化 & 0min30s63ms \\
		8 & 8 & 8 & 二分 & 0min1s252ms \\
		\bottomrule
	\end{tabular}
	\caption{算法运行效率对比}
	\label{my-label}
\end{table}

可见通过Z3内部的优化复杂度远小于二分法的O(logN)。

\subsection{运算时间增长趋势}
\begin{table}[!htp]
	\centering
	\begin{tabular}{@{}lllll@{}}
		\toprule
		行数 & 列数 & 终端对数 & 算法 & 时间 \\ \midrule
		3 & 3 & 3 & 内置优化 & 0min0s9ms \\
		4 & 4 & 4 & 内置优化 & 0min0s26ms \\
		5 & 5 & 5 & 内置优化 & 0min0s61ms \\
		6 & 6 & 6 & 内置优化 & 0min0s142ms \\
		7 & 7 & 7 & 内置优化 & 0min0s308ms \\
		8 & 8 & 8 & 内置优化 & 0min1s321ms \\
		9 & 9 & 9 & 内置优化 & 0min1s202ms \\
		10 & 10 & 10 & 内置优化 & 0min4s269ms \\
		11 & 11 & 11 & 内置优化 & 0min58s926ms \\
		12 & 12 & 12 & 内置优化 & 1min30s286ms \\
		\bottomrule
	\end{tabular}
	\caption{算法时间随地图大小增长}
	\label{my-label}
\end{table}

\begin{table}[!htp]
	\centering
	\begin{tabular}{@{}lllll@{}}
		\toprule
		行数 & 列数 & 终端对数 & 算法 & 时间 \\ \midrule
		9 & 9 & 9 & 内置优化 & 0min1s68ms \\
		9 & 9 & 10 & 内置优化 & 0min1s249ms \\
		9 & 9 & 11 & 内置优化 & 0min2s598ms \\
		9 & 9 & 12 & 内置优化 & 0min3s414ms \\
		9 & 9 & 13 & 内置优化 & 0min9s968ms \\
		9 & 9 & 14 & 内置优化 & 0min11s428ms \\
		9 & 9 & 15 & 内置优化 & 0min19s321ms \\
		9 & 9 & 16 & 内置优化 & 0min21s300ms \\
		9 & 9 & 17 & 内置优化 & 0min16s925ms \\
		9 & 9 & 18 & 内置优化 & 0min24s924ms \\
		9 & 9 & 19 & 内置优化 & 0min45s190ms \\
		9 & 9 & 20 & 内置优化 & 1min43s530ms \\
		9 & 9 & 21 & 内置优化 & 2min16s237ms \\
		9 & 9 & 22 & 内置优化 & 6min1s502ms \\
		\bottomrule
	\end{tabular}
	\caption{算法时间随终端对数增长}
	\label{my-label}
\end{table}

从上面两张表中可以看出,代码运行时间同时受到地图规模和终端对数所影响,任意一个较小都会有很高的速度。
\end{document}
